\section{The Binomial Series}
  The \textbf{Binomial Theorem} states that if $a$ and $b$ are any real numbers and $k$ is a positive integer, then
  \begin{equation*}
    \begin{split}
      (a+b)^k = a^k &+ ka^{k-1}b + \frac{k(k-1)}{2!}a^{k-2}b^2 + \frac{k(k-1)(k-2)}{3!}a^{k-3}b^3 \\
      &+ \cdots + \frac{k(k-1)(k-2) \cdots (k-n+1)}{n!}a^{k-n}b^n \\
      &+ \cdots + kab^{k-1}+b^k
    \end{split}
  \end{equation*}
  The traditional notation for the binomial coefficients is
  $$ \binom{k}{0} = 1 \qquad \binom{k}{n} = \frac{k(k-1)(k-2) \cdots (k-n+1)}{n!} \qquad n=1,2,\ldots,k $$
  which enables us to write the Binomial Theorem in the abbreviated form
  $$ (a+b)^k = \sum_{n=0}^{k} \binom{k}{n} a^{k-n}b^n $$
  \begin{definition}[\textbf{The Binomial Series}]
    If $k$ is any real number and $|x|<1$, then
    \begin{align*}
      (1+x)^k &= 1 + kx + \frac{k(k-1)}{2!}x^2 + \frac{k(k-1)(k-2)}{3!}x^3 + \cdots \\
      &= \sum_{n=0}^{\infty} \binom{k}{n} x^n
    \end{align*}
    where $\displaystyle\binom{k}{n} = \frac{k(k-1) \cdots (k-n+1)}{n!} \quad (n \geq 1)$ and $\displaystyle\binom{k}{0} = 1$
  \end{definition}
  \begin{proof}\let\qed\relax
    Newton extended the Binomial Theorem to the case in which $k$ is no longer a positive integer. In particular, if we put $a=1$ and $b=k$, we get
    $$(1+x)^k = \sum_{n=0}^{\infty} \binom{k}{n} x^n$$
    To find this series we compute the Maclaurin series of $(1+x)^k$ in the usual way.
    \begin{equation*}
      \begin{aligned}[c]
        f(x) &= (1+x)^k \\
        f'(x) &= k(1+x)^{k-1} \\
        f''(x) &= k(k-1)(1+x)^{k-2} \\
        f'''(x) &= k(k-1)(k-2)(1+x)^{k-3} \\
        \vdots \\
        f^{(n)}(x) &= k(k-1) \cdots (k-n+1)(1+x)^{k-n}
      \end{aligned}
      \qquad
      \begin{aligned}[c]
        f(0) &= 1 \\
        f'(0) &= k \\
        f''(0) &= k(k-1) \\
        f'''(0) &= k(k-1)(k-2) \\
        \vdots \\
        f^{(n)}(0) &= k(k-1) \cdots (k-n+1)
      \end{aligned}
    \end{equation*}
    Therefore the Maclaurin series of $f(x) = (1+x)^k$ is
    $$\sum_{n=0}^{\infty} \frac{f^{(n)}(0)}{n!}x^n = \sum_{n=0}^{\infty} \frac{k(k-1) \cdots (k-n+1)}{n!}x^n$$
    Now we use the Ratio Test to test the binomial series for convergence. If the $n$th term is $a_n$, then
    \begin{align*}
      \left| \frac{a_{n+1}}{a_n} \right| &= \left| \frac{k(k-1) \cdots (k-n+1)(k-n)x^{n+1}}{(n+1)!} \cdot \frac{n!}{k(k-1) \cdots (k-n+1)x^n} \right| \\
      &= \frac{|k-n|}{x+1}|x| = \frac{\left| 1 - \dfrac{k}{n} \right|}{ 1 + \dfrac{1}{n}}|x| \to |x| \quad \text{as } n\to\infty
    \end{align*}
    The binomial series converges if $|x|<1$ and diverges if $|x|>1$ by the Ratio Test.
  \end{proof}
  \begin{example}
    Expand $\dfrac{1}{(1+x)^2}$ as a power series.
  \end{example}
  \begin{solution}
    Use the binomial series with $k=2$. The binomial coefficient is
    \begin{align*}
      \binom{-2}{n} &= \frac{(-2)(-3)(-4)\cdots(-2-n+1)}{n!} \\
      &= \frac{(-1)^n 2\cdot3\cdot4\cdot\ldots\cdot n(n+1)}{n!} = (-1)^n(n+1)
      \intertext{and so, when $|x|<1$,}
      \frac{1}{(1+x)^2} &= (1+x)^{-2} = \sum_{n=0}^{\infty} \binom{-2}{n} x^n \\
      &= \sum_{n=0}^{\infty} (-1)^n(n+1)x^n = 1-2x+3x^2-4x^3 + \cdots
    \end{align*}
  \end{solution}