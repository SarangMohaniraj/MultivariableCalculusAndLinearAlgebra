\section{Absolute Convergence and the Ratio and Root Tests}
  Given any series $\sum a_n$, we can consider the corresponding series
  $$ \sum_{n=1}^{\infty} |a_n| = |a_1|+|a_2|+|a_3|+\cdots$$
  whose terms are the absolute value of the terms of the original series.
  \begin{definition}
    A series $\sum a_n$ is called \textbf{absolutely convergent} if the series of absolute values $\sum |a_n|$ is convergent.
  \end{definition}
  Notice that if $\sum a_n$ is a series with positive terms, then $\sum |a_n|$ = $\sum a_n$ and so absolute convergence is the same as convergence in this case.
  \begin{example}
    The series
    $$ \sum_{n\to\infty}^{\infty} \frac{(-1)^{n-1}}{n^2} = 1 - \frac{1}{2^2} + \frac{1}{3^2} - \frac{1}{4^2} + \cdots$$
    is absolutely convergent because
    $$ \sum_{n\to\infty}^{\infty} \left|\frac{(-1)^{n-1}}{n^2}\right| = 1 + \frac{1}{2^2} + \frac{1}{3^2} + \frac{1}{4^2} + \cdots$$
     is a convergent $p$-series $(p=2)$.
  \end{example}
  \begin{definition}
    A series $\sum a_n$ is \textbf{conditionally convergent} if it is convergent but not absolutely convergent.
  \end{definition}
  \begin{definition}
    If a series $\sum a_n$ is absolutely convergent, then it is convergent.
  \end{definition}
  \begin{proof}\let\qed\relax
    Observe that the inequality $$0 \leq a_n + |a_n| \leq 2|a_n|$$ is true because $|a_n|$ is either $a_n$ or $-a_n$. If $\sum a_n$ is absolutely convergent, then $\sum |a_n|$ is convergent, $\sum 2|a_n|$. Therefore, by the comparison test, $\sum (a_n + |a_n|)$ is convergent. Then $$\sum a_n = \sum (a_n + |a_n|) - \sum |a_n|$$ is the difference of the two convergent series and is therefore convergent.
  \end{proof}
  \begin{minipage}{\linewidth}
    \begin{definition}
      \textbf{The Ratio Test}\\
      \begin{enumerate}
        \item[(i)] If $\displaystyle \lim_{n\to\infty} \left|\frac{a_{n+1}}{a_n}\right| = L < 1$, then the series $\displaystyle \sum_{n=1}^{\infty}  a_n$ is absolutely convergent (and therefore convergent).
        \item[(ii)] If $\displaystyle \lim_{n\to\infty} \left|\frac{a_{n+1}}{a_n}\right| = L > 1$ or If $\displaystyle \lim_{n\to\infty} \left|\frac{a_{n+1}}{a_n}\right| = \infty$, then the series $\displaystyle \sum_{n=1}^{\infty}  a_n$ is divergent.
        \item[(iii)] If $\displaystyle \lim_{n\to\infty} \left|\frac{a_{n+1}}{a_n}\right| = 1$, the Ratio Test is inconclusive (no conclusion can be drawn about the convergence or divergence of $\sum a_n$).
      \end{enumerate}
    \end{definition}
  \end{minipage}
  \begin{proof}\let\qed\relax
    \hphantom{ }\\~\\
    \begin{enumerate}
        \item[(i)] The idea is to compare the given series with a convergent geometric series. Since $L>1$, we can choose a number $r$ such that $L<r<1$. Since
        $$\lim_{n\to\infty} \left|\frac{a_{n+1}}{a_n}\right| = L \quad\text{and}\quad L<r $$
        the ratio $|a_{n+1}/a_n|$ will eventually be less than $r$; this means that there exists an integer $N$ such that
        $$ \left|\frac{a_{n+1}}{a_n}\right| < r \quad\text{whenever  } n \geq N$$
        or, equivalently,
        $$ a_{n+1} < r|a_n| \quad\text{whenever  } n \geq N$$
        Putting $n$ successively equal to $N,\ N+1,\ N+2,\ldots$ in the previous equation, we obtain
        \begin{align*}
          |a_{N+1}| &< |a_{N}|r \\
          |a_{N+2}| &< |a_{N+1}|r < |a_N|r^2 \\
          |a_{N+3}| &< |a_{N+2}|r < |a_N|r^3
        \end{align*}
        and, in general,
        $$  |a_{N+k}| < |a_N|r^k \qquad\text{for all } k \geq 1 $$
        Now the series
        $$ \sum_{k=1}^{\infty} |a_N|r^k = |a_N|r + |a_N|r^2 + |a_N|r^3 + \cdots $$
        is convergent because it is a geometric series with $0 < r < 1$. So the previous inequality , together with the Comparison Test, shows that the series
        $$ \sum_{n=N+1}^{\infty} |a_n| = \sum_{k=1}^{\infty} |a_{N+k}| = |a_{N+1}| + |a_{N+2}| + |a_{N+3}| + \cdots $$
        is also convergent. It follows that the series $\sum_{n=1}^{\infty} |a_n|$ is also convergent. Therefore, $\sum a_n$ is absolutely convergent.
        \\~\\
        \item[(ii)] If $|a_{n+1}/a_n| \to L > 1$ or $|a_{n+1}/a_n| \to \infty$, then the ratio $|a_{n+1}/a_n|$ will eventually be greater than 1. This means that there exists an integer $N$ such that
        $$ \left|\frac{a_{n+1}}{a_n}\right| > 1 \quad\text{whenever  } n \geq N $$
        This means that $|a_{n+1}/a_n| > |a_n|$ whenever $n \geq N$ and so $$\lim_{n\to\infty}a_n \neq 0$$
        Therefore $\sum a_n$ diverges by the Test for Divergence.
        \\~\\
        \item[(iii)] The Ratio Test gives no information if $\lim_{n\to\infty} |a_{n+1}/a_n| = 1$. For instance,
         \\~\\
         for the convergent series $\sum 1/n^2$ we have
         $$ \left|\frac{a_{n+1}}{a_n}\right| = \frac{\dfrac{1}{(n+1)^2}}{\dfrac{1}{n^2}} = \frac{n^2}{(n+1)^2} = \frac{1}{\left(1+\dfrac{1}{n}\right)^2} \to 1 \quad\text{as } n\to\infty $$
         whereas for the divergent series $\sum 1/n$ we have
         $$ \left|\frac{a_{n+1}}{a_n}\right| = \frac{\dfrac{1}{n+1}}{\dfrac{1}{n}} = \frac{n}{n+1} = \frac{1}{1+\dfrac{1}{n}} \to 1 \quad\text{as } n\to\infty $$
         Therefore, if $\lim_{n\to\infty} |a_{n+1}/a_n| = 1$, the series $\sum a_n$ might converge or diverge. In this case, the Ratio Test fails and we must use some other test.
      \end{enumerate}
  \end{proof}
  \begin{example}
    Test the series $\displaystyle \sum_{n=1}^{\infty} (-1)^n \frac{n^3}{3^n} for absolute convergence.$
  \end{example}
  \begin{solution}
    We use the Ratio Test with $a_n = (-1)^n n^3/3^n$.
    \begin{align*}
      \left|\frac{a_{n+1}}{a_n}\right| &= \left|\frac{\dfrac{(-1)^{n+1}(n+1)^3}{3(n+1)}}{\dfrac{(-1)^n n^3}{3^n}}\right| = \frac{(n+1)^3}{3^{n+1}} \cdot \frac{3^n}{n^3} \\
      &= \frac{1}{3} \left(\frac{n+1}{n}\right)^3 = \frac{1}{3} \left(1 + \frac{1}{n}\right)^3 \to \frac{1}{3} < 1
    \end{align*}
    The given series is absolutely convergent by the Ratio Test and therefore convergent.
  \end{solution}
  \begin{definition}
    \textbf{The Root Test}\\
    \begin{enumerate}
      \item[(i)] If $\displaystyle\lim_{n\to\infty} \sqrt[n]{|a_n|} = L < 1$, then the series $\displaystyle\sum_{n=1}^{\infty} a_n$ is is absolutely convergent (and therefore convergent).
      \item[(ii)] If $\displaystyle\lim_{n\to\infty} \sqrt[n]{|a_n|} = L > 1$ or $\displaystyle\lim_{n\to\infty} \sqrt[n]{|a_n|} = \infty$, then the series $\displaystyle\sum_{n=1}^{\infty} a_n$ is divergent.
      \item[(iii)] If $\displaystyle\lim_{n\to\infty} \sqrt[n]{|a_n|} = 1$, then the Root Test is inconclusive.
    \end{enumerate}
    \textsc{Note} If $L=1$ in the Ratio Test, don't try the Root Test because $L$ will also be 1.
  \end{definition}
  \begin{example}
    Test the convergence of the series $\displaystyle \sum_{n=1}^{\infty} \left(\frac{2n+3}{3n+2}\right)^n$
  \end{example}
  \begin{solution}
    \begin{align*}
      a_n &= \left(\frac{2n+3}{3n+2}\right)^n \\
      \sqrt[n]{|a_n|} &= \frac{2n+3}{3n+2} = \frac{2 + \dfrac{3}{n}}{3 + \dfrac{2}{n}} \to \frac{2}{3} < 1
    \end{align*}
    The given series converges by the Root Test.
  \end{solution}
