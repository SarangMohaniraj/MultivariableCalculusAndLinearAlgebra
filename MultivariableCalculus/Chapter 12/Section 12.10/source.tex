\section{Taylor and Maclaurin Series}
  \begin{theorem}
    If $f$ has a power series representation (expansion) at $a$, that is, if
    $$ f(x) = \sum_{n=0}^{\infty} c_n (x-a)^n \qquad |x-a|<R $$
    then its coefficients are given by the formula
    $$ c_n = \frac{f^{(n)}(a)}{n!} $$
  \end{theorem}
  Substituting this formula for $c_n$ back into the series, we see that \textit{if} $f$ has a power series expansion at $a$, then it must be of the following form.
  \begin{definition}
    The \textbf{Taylor series of the function $f$ at $a$} (or \textbf{about $a$} or \textbf{centered at $a$}) is of the form
    \begin{align*}
      f(x) &= \sum_{n=0}^{\infty} \frac{f^{(n)}(a)}{n!} (x-a)^n \\
      &= f(a) + \frac{f'(a)}{1!}(x-a) + \frac{f''(a)}{2!}(x-a)^2 + \frac{f'''(a)}{3!}(x-a)^3 + \cdots
    \end{align*}
  \end{definition}
  \begin{definition}
    The \textbf{Maclaurin series} is a special case of the Taylor series when $a=0$.
    $$ f(x) = \sum_{n=0}^{\infty} \frac{f^{(n)}(0)}{n!} x^n = f(0) + \frac{f'(0)}{1!}x + \frac{f''(0)}{2!}x^2 + \cdots $$
  \end{definition}
  \begin{example}
    Find the Maclaurin series of the function $f(x)=e^x$ and its radius of convergence. Also, find the Taylor series at $a$.
  \end{example}
  \begin{solution}
    If $f(x)=e^x$, then $f^{n}(x)=e^x$, so $f^{n}(0)=e^x = e^0 = 1$ for all $n$. Therefore, the Taylor series for $f$ at 0 (Maclaurin series) is
    $$\sum_{n=0}^{\infty} \frac{f^{(n)}(0)}{n!} = \sum_{n=0}^{\infty} \frac{x^n}{n!} $$
    To find the radius of convergence we let $a_n = x^n/n!$, so
    $$ \left| \frac{a_{n+1}}{a_n} \right| = \left| \frac{x^{n+1}}{(n+1)!} \cdot \frac{n!}{x^n}\right| = \frac{|x|}{n+1} \to 0 < 1 $$
    so the series converges for all $x$ by the Ratio Test and the radius of convergence is $R=\infty$.
    \\~\\
    The Taylor series at $a$ is $$f(x) = \sum_{n=0}^{\infty} \frac{f^{(n)}(a)}{n!}(x-a)^n = \sum_{n=0}^{\infty} \frac{e^a}{n!}(x-a)^n $$
  \end{solution}
  \begin{theorem}
    If $f(x) = T_n (x)+ R_n (x)$, where $T_n$ is the $n$th degree polynomial if $f$ at $a$, $R_n$ is the remainder of the Taylor series, and
    $$\lim_{n\to\infty} R_n (x) = 0 $$
    for $|x-a|<R$, then $f$ is equal to the sum of its Taylor series on the interval $|x-a|<R$.
  \end{theorem}
  \begin{proof}\let\qed\relax
    We are determining under what circumstances is a function equal to the sum of its Taylor Series. In other words, if $f$ has derivatives of all orders, when is it true that
    $$f(x) = \sum_{n=0}^{\infty} \frac{f^{(n)}(a)}{n!}(x-a)^n $$
    As with any convergent series, this means that $f(x)$ is the limit of the sequence of partial sums. In the case of the Taylor series, the partial sums are
    \begin{align*}
      T_n (x) &= \sum_{i=0}^{\infty} \frac{f^{(i)}(a)}{n!}(x-a)^i \\
      &= f(a) + \frac{f'(a)}{1!}(x-a) + \frac{f''(a)}{2!}(x-a)^2 + \cdots + \frac{f^{(n)}(a)}{n!}(x-a)^n
    \end{align*}
    For example, for the polynomial function $f(x) = e^x$, the Taylor polynomials at 0 (or Maclaurin polynomials) with $n=1,\ 2,\ \text{and } 3$ are
    $$ T_1 (x)=1+x \qquad T_2 (x)=1+x + \frac{x^2}{2!} \qquad T_3 (x)=1+x +\frac{x^2}{2!} +\frac{x^3}{3!} $$
    In general, $f(x)$ is the sum of its Taylor series if $$f(x) = \lim_{n\to\infty} T_n (x) $$
    If we let $$ R_n (x) = f(x) - T_n (x) \quad\text{so that}\quad f(x) = T_n (x) + R_n (x) $$
    then $R_n (x)$ is the remainder of the Taylor series. If we can somehow show that $\lim_{n\to\infty} R_n (x) = 0 $, then it follows that
    $$\lim_{n\to\infty} T_n (x) = \lim_{n\to\infty} [f(x) - R_n (x)]  = f(x) - \lim_{n\to\infty} R_n (x) = f(x) $$
  \end{proof}
  In trying to show that $\lim_{n\to\infty} R_n (x) = 0 $ for a specific function $f$, we usually use the following fact.
  \begin{definition}[\textbf{Taylor's Inequality}]
    If $|f^{(n+1)}(x)| \leq M$ for $|x-a| \leq d$, then the remainder $R_n (x)$ of the Taylor series satisfies the inequality
    $$|R_n (x)| \leq \frac{M}{(n+1)!}|x-a|^{n+1} \qquad\text{for } |x-a| \leq d $$
  \end{definition}
  It is helpful to use the following fact.
  \begin{definition}
    $\displaystyle\lim_{n\to\infty} \frac{x^n}{n!} = 0$ \quad for every real number $x$
  \end{definition}
  \begin{definition}
   $$\displaystyle  e^x = \sum_{n=0}^{\infty} \frac{x^n}{n!} \quad \text{for every real number } x$$
   If we plug in $x=1$, we get the following expression for the number $e$ as a sum of an infinite series:
   $$ e = \sum_{n=0}^{\infty} \frac{1}{n!} = 1 + \frac{1}{1!} + \frac{2}{2!} + \frac{3}{3!} + \cdots $$
  \end{definition}
  \begin{proof}\let\qed\relax
    If $f(x) = e^x$, then $f^{(n+1)}(x) = e^x$ for all $n$. If $d$ is a positive number and $|x| \leq d$, then $|f^{(n+1)}(x)| = e^x \leq e^d$. So Taylor's inequality, with $a=0$, and $M=e^d$, says that
    $$ |R_n (x)| \leq \frac{e^d}{(n+1)!}|x|^{n+1} \qquad \text{for} |x| \leq d $$
    Notice that we have the same constant $M = e^d$ for every value of $n$. But because $\displaystyle\lim_{n\to\infty} \frac{x^n}{n!} = 0$, we have
    $$ \lim_{n\to\infty} \frac{e^d}{(n+1)!}|x|^{n+1} = e^d \lim_{n\to\infty} \frac{|x|^{n+1}}{(n+1)!} = 0 $$
    It follows from the Squeeze Theorem that $\lim_{n\to\infty} |R_n (x)| = 0$ and therefore $\lim_{n\to\infty} R_n (x) = 0$ for all values of $x$.
  \end{proof}
  \begin{definition}
    \hphantom{ } \\
    \begin{enumerate}
      \item[(i)] $\displaystyle \sin x = \sum_{n=0}^{\infty} (-1)^n \frac{x^{2n+1}}{(2n+1)!}$ \qquad for all $x$
      \item[(ii)] $\displaystyle \cos x = \sum_{n=0}^{\infty} (-1)^n \frac{x^{2n}}{(2n)!}$ \qquad for all $x$
    \end{enumerate}
  \end{definition}
  \begin{proof}\let\qed\relax
    The strategy is to find the Maclaurin series for $\sin x$ and prove that it represents $\sin x$ for all $x$, and then differentiate it to find the Maclaurin series for $\cos x$.
    \begin{enumerate}
      \item[(i)] We arrange our computation in two columns: \\
      \begin{minipage}{\textwidth}
        \begin{center}
          \bgroup
          \def\arraystretch{1.5}
          \begin{tabular}{ r r }
            $f(x) = \sin x$  & $f(0)= 0$ \\
            $f'(x) = \cos x$  & $f'(0)= 1$ \\
            $f''(x) = -\sin x$  & $f''(0)= 0$ \\
            $f'''(x) = -\cos x$  & $f'''(0)= -1$ \\
            $f^{(4)}(x) = \sin x$  & $f^{(4)}(0)= 0$ \\
          \end{tabular}
          \egroup
        \end{center}
      \end{minipage}
      Since the derivatives repeat in a cycle of four, we can write the Maclaurin series as
      \begin{align*}
        f(0) &+ \frac{f'(0)}{1!} + \frac{f''(0)}{2!} + \frac{f'''(0)}{3!} + \cdots \\
        &= x - \frac{x^3}{x!} + \frac{x^5}{5!} - \frac{x^7}{7!} + \cdots \\
        &= \sum_{n=0}^{\infty} (-1)^n \frac{x^{2n+1}}{(2n+1)!}
      \end{align*}
      Since $f^{(n+1)}(x)$ is $\pm \sin x$ or $\pm \cos x$, we know that $|f^{(n+1)}(x)| <\leq 1$ for all $x$, so we can take $M=1$ in Taylor's inequality:
      $$ |R_n (x)| \leq \frac{M}{(n+1)!}|x|^{n+1} = \frac{|x|^{n+1}}{(n+1)!} $$
      Since $\displaystyle\lim_{n\to\infty} \frac{x^n}{n!} = 0$, the right side of this inequality approaches 0 as $n\to\infty$, so $|R_n (x)| \to 0$ by the Squeeze Theorem. It follows that $R_n (x) \to 0$ as $n\to\infty$, so $\sin x$ is equal to the sum of its Maclaurin series.
      \item[(ii)] We could proceed directly as the previous proof but it is easier to differentiate the Maclaurin series for $\sin x$.
      \begin{align*}
        \cos x &= \frac{d}{dx} \sin x = \frac{d}{dx} \left( x - \frac{x^3}{x!} + \frac{x^5}{5!} - \frac{x^7}{7!} + \cdots \right) \\
        &= 1- \frac{3x^2}{3!} + \frac{5x^4}{5!} - \frac{7x^6}{7!} + \cdots = 1- \frac{x^2}{2!} + \frac{x^4}{4!} - \frac{x^6}{6!} + \cdots \\
        \intertext{Since the Maclaurin series for $\sin x$ converges for all $x$, the differentiated series for $\cos x$ also converges for all $x$, so}
        &= \sum_{n=0}^{\infty} (-1)^n \frac{x^{2n}}{(2n)!} \qquad \text{for all } x
      \end{align*}
    \end{enumerate}
  \end{proof}
  \begin{example}
    Find the Maclaurin series for the function $f(x)=x\cos x$.
  \end{example}
  \begin{solution}
    Multiply the series for $\cos x$ by $x$.
    $$ x\cos x = x \sum_{n=0}^{\infty} (-1)^n \frac{x^{2n}}{(2n)!} = \sum_{n=0}^{\infty} (-1)^n \frac{x^{2n+1}}{(2n)!} $$
  \end{solution}
  \begin{minipage}{\textwidth}
    \begin{center}
      \bgroup
      \def\arraystretch{3}
      \begin{tabular}{ |l c| }
        \hline
        Important Maclaurin series & Interval of Convergence \\
        \hline
        $\displaystyle \frac{1}{1-x} = \sum_{n=0}^{\infty} x^n = 1 + x + x^2 + x^3 + \cdots$ & $(-1,1)$ \\
        $\displaystyle e^x = \sum_{n=0}^{\infty} \frac{x^n}{n!} = 1 + \frac{x}{1!} + \frac{x^2}{2!} + \frac{x^3}{3} + \cdots$ & $(-\infty,\infty)$ \\
        $\displaystyle \sin x = \sum_{n=0}^{\infty} (-1)^n \frac{x^{2n+1}}{(2n+1)!} = x - \frac{x^3}{x!} + \frac{x^5}{5!} - \frac{x^7}{7!} + \cdots$ & $(-\infty,\infty)$ \\
        $\displaystyle \cos x = \sum_{n=0}^{\infty} (-1)^n \frac{x^{2n}}{(2n)!} = 1- \frac{x^2}{2!} + \frac{x^4}{4!} - \frac{x^6}{6!} + \cdots $ & $(-\infty,\infty)$ \\
        $\displaystyle \tan^{-1}x = \sum_{n=0}^{\infty} (-1)^n \frac{x^{2n+1}}{2n+1} = x - \frac{x^3}{x!} + \frac{x^5}{5!} - \frac{x^7}{7!} + \cdots$ & $(-1,1]$ \\
        \hline
      \end{tabular}
      \egroup
    \end{center}
  \end{minipage}
  \\~\\
  \begin{example}
    \hphantom{ }\\
    \begin{enumerate}
      \item[(a)] Evaluate $\displaystyle \int e^{-x^2}\ dx$ as an infinite series.
      \item[(b)] Evaluate $\displaystyle \int_{0}^{1} e^{-x^2}\ dx$ correct to within an error of 0.001.
    \end{enumerate}
  \end{example}
  \begin{solution}
    \hphantom{ }\\
    \begin{enumerate}
      \item[(a)] First we find the Maclaurin series for $e^{-x^2}\ dx$ by replacing $x$ with $-x^2$ in the series for $e^x$ given in the table for Maclaurin series. Thus, for all values of $x$,
      $$ e^{-x^2} = \sum_{n=0}^{\infty} \frac{(-x^2)^n}{n!}  = \sum_{n=0}^{\infty} (-1)^n \frac{x^{2n}}{n!} = 1 - \frac{x^2}{1!} + \frac{x^4}{2!} - \frac{x^6}{3} + \cdots$$
      Now integrate by term:
      \begin{align*}
        \int e^{-x^2}\ dx &= \int \left( 1 - \frac{x^2}{1!} + \frac{x^4}{2!} - \frac{x^6}{3} + \cdots \right)\ dx \\
        &= C + x - \frac{x^3}{3 \cdot 1!} + \frac{x^5}{5 \cdot 2!} - \frac{x^7}{7 \cdot 3!} + \cdots + (-1)^n \frac{x^{2n+1}}{(2n+1)n!} + \cdots
      \end{align*}
      The series converges for all $x$ because the original series $e^{-x^2}$ converges for all $x$.
      \item[(b)]
      \begin{align*}
        \int_{0}^{1} e^{-x^2}\ dx &= \left[ x - \frac{x^3}{3 \cdot 1!} + \frac{x^5}{5 \cdot 2!} - \frac{x^7}{7 \cdot 3!} + \frac{x^9}{9 \cdot 3!} - \cdots \right]_{0}^{1} \\
        &= 1 - \tfrac{1}{3} + \tfrac{1}{10} - \tfrac{1}{42} + \tfrac{1}{216} - \cdots \\
        &\approx 1 - \tfrac{1}{3} + \tfrac{1}{10} - \tfrac{1}{42} + \tfrac{1}{216} \approx 0.7475
      \end{align*}
      The Alternating Series Estimation THeorem shows that the error involved in this approximation is less than
      $$ \frac{1}{11 \cdot 5!} = \frac{1}{1320} < 0.001 $$
    \end{enumerate}
  \end{solution}
  \subsection*{Multiplication and Division of Power Series}
  If power series are added or subtracted they behave like polynomials. In fact, they can also be multiplied and divided like polynomials. We only find the first few terms because they are the most important ones and the calculations for the later terms become tedious.
  \begin{example}
    Find the first three nonzero terms in the Maclaurin series for (a) $e^x \sin x$ and (b) $\tan x$.
  \end{example}
  \begin{solution}
    \hphantom{ }\\
    \begin{enumerate}
      \item[(a)] Using the Maclaurin series for $e^x$ and $\sin x$ from the table,
      $$e^x \sin^x = \left( 1 + \frac{x}{1!} + \frac{x^2}{2!} + \frac{x^3}{3} + \cdots \right) \left( x - \frac{x^3}{x!} + \cdots \right)$$
      We multiply these expressions like polynomials and get
      $$e^x \sin^x = x + x^2 + \tfrac{1}{3}x^3 + \cdots$$
      \item[(b)] Using the Maclaurin series in the table, we have
      $$\tan x =\frac{\sin x}{\cos x} = \frac{x - \frac{x^3}{3!} + \frac{x^5}{5!} - \cdots}{1 - \frac{x^2}{2!} + \frac{x^4}{4!} - \cdots} $$
      Use long-division to get
      $$\tan x = x + \tfrac{1}{3}x^3 + \tfrac{2}{15}x^5 + \cdots $$
    \end{enumerate}
  \end{solution}