\section{Applications of Taylor Polynomials}
  Two applications of Taylor polynomials are to approximate functions and use them in physics.
  \subsection*{Approximating Functions by Polynomials}
    In section 12.10, we introduced the notion for $T_n (x)$ for the $n$th partial sum of this series. Since $f$ is the sum of its Taylor series, we know that $T_n (x) \to \infty$ as $n\to\infty$ so $T_n$ can be used as an approximation to $f$: $f(x) \approx T_n (x)$.\par
    When using a Taylor polynomial $T_n$ to approximate a function $f$, we have to ask the questions: How good an approximation is it? How large should we take n to be in order to achieve a desired accuracy? To answer these questions we need to look at the absolute value of the remainder:
    $$|R_n (x)| = |f(x) - T_n (x)|$$
    There are three possible methods for estimating the size of the error:
    \begin{enumerate}
      \item If a graphing device is available, we can use it to graph $|R_n (x)|$ and thereby graphing the error.
      \item If the series happens to be an alternating series, we can use the Alternating Series Estimation Theorem.
      \item In all cases we can use Taylor's Inequality, which says that if $|f^{(n+1)}(x)| \leq M$, then
      $$|R_n (x)| \leq \frac{M}{(n+1)!}|x-a|^{n+1}$$
    \end{enumerate}
    \begin{example}
      \hphantom{ }\\
      \begin{enumerate}
        \item[(a)] Approximate the function $f(x) = \sqrt[3]{x}$ by a Taylor polynomial of degree 2 at $a=8$.
        \item[(b)] How accurate is this approximation when $7 \leq x \leq 9$?
      \end{enumerate}
    \end{example}
    \begin{solution}
      \hphantom{ }\\
      \begin{enumerate}
        \item[(a)]
          \begin{equation*}
            \begin{aligned}[c]
              f(x) &= \sqrt[3]{x}= x^{1/3} \\
              f'(x) &= \tfrac{1}{3}x^{-2/3} \\
              f''(x) &= -\tfrac{2}{9}x^{-5/3} \\
              f'''(x) &= \tfrac{10}{27}x^{-8/3}
            \end{aligned}
            \qquad\qquad
            \begin{aligned}[c]
              f(8) &= 2 \\
              f'(8) &= \tfrac{1}{12} \\
              f'(8) &= -\tfrac{1}{144} \\
              \hphantom{ }
            \end{aligned}
          \end{equation*}
          \begin{align*}
            \intertext{Thus, the second-degree Taylor polynomial is}
            T_2 (x) &= f(8) + \frac{f'(8)}{1!}(x-8) + \frac{f'(8)}{2!}(x-8)^2 \\
            &= 2 + \tfrac{1}{12}(x-8) - \tfrac{1}{288}(x-8)^2
            \intertext{The desired approximation is}
            \sqrt[3]{x} &\approx T_2 (x) = 2 + \tfrac{1}{12}(x-8) - \tfrac{1}{288}(x-8)^2
          \end{align*}

        \item[(b)] The Taylor series is not alternating when $x<8$, so we can't use the Alternating Series Estimation Theorem, but we can use Taylor's Inequality with $n=2$ and $a=8$:
        $$|R_2 (x)| \leq \frac{M}{3!}|x-8|^{3}$$
        where $|f'''(x)| \leq M$. Because $x \geq 7$, we have $x^{8/3} \geq 7^{8/3}$ and so
        $$f'''(x) = \frac{10}{27} \cdot \frac{1}{x^{8/3}} \leq \frac{10}{27} \cdot \frac{1}{7^{8/3}} < 0.0021$$
        Therefore, we can take $M=0.0021$. Also $7 \leq x \leq 9$ so $-1 \leq x-8 \leq 1$ and $|x-8| \leq 1$. Taylor's inequality gives
        $$|R_2 (x)| \leq \frac{0.0021}{3!}\cdot 1^3 = \frac{0.0021}{6} \leq 0.0004$$
        Thus, if $7 \leq x \leq 9$, the approximation in part (a) is accurate to within 0.0004.
      \end{enumerate}
    \end{solution}