\section{Areas and Lengths in Polar Coordinates}
  \subsection*{Area}
    We can determine the formula for the area of a region whose boundary is given by a polar equation by taking the limit of a Riemann Sum starting with the formula for the area of a sector of a circle $A=\frac{1}{2}r^2 \theta$.
    \begin{definition}
      The formula for the area A of the polar region $\mathcal{R}$ is
      $$A=\int_{a}^{b}\frac{1}{2}[f(\theta)]^2\ d\theta = \int_{a}^{b}\frac{1}{2}r^2 d\theta$$
      with the understanding that $r=f(\theta)$.
    \end{definition}
    \begin{example}
      Find the area enclosed by one loop of the four-leaved rose $r=2\cos2\theta$.
    \end{example}
    \begin{solution}
      The right loop rotates from $\theta=-\pi/4$ to $\theta=\pi/4$.
      \begin{align*}
        A &= \int_{-\pi/4}^{\pi/4}\frac{1}{2}r^2 d\theta = \frac{1}{2}\int_{-\pi/4}^{\pi/4}\cos^2 2\theta\ d\theta \\
          &= \int_{0}^{\pi/4}\cos^2 2\theta\ d\theta = \int_{0}^{\pi/4}\frac{1}{2}(1+\cos4\theta)\ d\theta \\
          &= \frac{1}{2}[\theta+\frac{1}{4}\sin4\theta] = \pi/8
      \end{align*}
    \end{solution}
    We can also adapt the formula to find the area of a region bounded by two polar curves.
    \begin{definition}
      Let $\mathcal{R}$ be a region that is bounded by curves with polar equations $r=f(\theta)$, $r=g(\theta)$, $\theta=a$, and $\theta=b$, where $f(\theta) \geq g(\theta) \geq 0$ and $0 < b-a \leq 2\pi$. The area $A$ of $\mathcal{R}$ is found by subtracting the area inside $r=g(\theta)$ from the area inside $r=f(\theta)$, so
      \begin{align*}
        A &= \int_{a}^{b}\frac{1}{2}[f(\theta)]^2\ d\theta - \int_{a}^{b}\frac{1}{2}[g(\theta)]^2\ d\theta \\
          &= \int_{a}^{b}\frac{1}{2}([f(\theta)]^2 - [g(\theta)]^2)\ d\theta
      \end{align*}
    \end{definition}
  \subsection*{Arc Length}
    To find the length of a polar curve $r=f(\theta)$, $a\leq\theta\leq b$, we regard $\theta$ as a parameter and write the parametric equations of the curve as
    $$x=r\cos\theta=f(\theta)\cos\theta \ \ \ \ y=r\sin\theta=f(\theta)\sin\theta$$
    Using the projecut Rule and differentiating with respect to $\theta$, we obtain
    $$\frac{dx}{d\theta}=\frac{dr}{d\theta}\cos\theta-r\sin\theta \ \ \ \ \frac{dy}{d\theta}=\frac{dr}{d\theta}\sin\theta+r\cos\theta$$
    so, using $\cos^2 \theta + \sin^2 \theta = 1$, we have
    \begin{align*}
      \left(\frac{dx}{d\theta}\right)^2 + \left(\frac{dy}{d\theta}\right)^2 &= \left(\frac{dr}{d\theta}\right)^2 \cos^2 \theta - 2r\frac{dr}{d\theta}\cos\theta\sin\theta+r^2 \sin^2 \theta \\
      &+ \left(\frac{dr}{d\theta}\right)^2 \sin^2 \theta + 2r\frac{dr}{d\theta}\sin\theta\cos\theta+r^2 \cos^2 \theta \\
      &= \left(\frac{dr}{d\theta}\right)^2 + r^2
    \end{align*}
    Assuming that $f'$ is continuous, we can use the theorem from 11.2 about the arc length of a curve defined by parametric equations to write the arc length as
    $$L=\int_{a}^{b}\sqrt{\left(\frac{dx}{d\theta}\right)^2 + \left(\frac{dy}{d\theta}\right)^2}\ d\theta$$
    \begin{definition}
      The length of a curve with polar equation $r=f(\theta),\ a\leq\theta\leq b$, is
      $$L=\int_{a}^{b}\sqrt{r^2 + \left(\frac{dr}{d\theta}\right)^2}\ d\theta$$
    \end{definition}
    \begin{example}
      Find the arc length of the cardiod $r=1+\sin\theta$.
    \end{example}
    \begin{solution}
      The full length of the cardiod is given by the parameter interval $0\leq\theta\leq2\pi$.
      \begin{align*}
        L &= \int_{0}^{2\pi}\sqrt{r^2 + \left(\frac{dr}{d\theta}\right)^2}\ d\theta = \int_{0}^{2\pi}\sqrt{(1+\sin\theta)^2 + \cos^2 \theta}\ d\theta \\
          &= \int_{0}^{2\pi}\sqrt{2+2\sin\theta}\ d\theta = 8\ (\textit{by rationalizing the integrand by $\sqrt{2-2\sin\theta}$})
      \end{align*}
    \end{solution}